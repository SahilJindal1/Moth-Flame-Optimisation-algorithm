\documentclass[12pt,a4paper]{report}
%\usepackage{graphicx}
\usepackage{physics}
\usepackage{amsmath}


    \usepackage[breakable]{tcolorbox}
    \usepackage{parskip} % Stop auto-indenting (to mimic markdown behaviour)
    
    \usepackage{iftex}
    \ifPDFTeX
    	\usepackage[T1]{fontenc}
    	\usepackage{mathpazo}
    \else
    	\usepackage{fontspec}
    \fi

    % Basic figure setup, for now with no caption control since it's done
    % automatically by Pandoc (which extracts ![](path) syntax from Markdown).
    \usepackage{graphicx}
    % Maintain compatibility with old templates. Remove in nbconvert 6.0
    \let\Oldincludegraphics\includegraphics
    % Ensure that by default, figures have no caption (until we provide a
    % proper Figure object with a Caption API and a way to capture that
    % in the conversion process - todo).
    \usepackage{caption}
    \DeclareCaptionFormat{nocaption}{}
    \captionsetup{format=nocaption,aboveskip=0pt,belowskip=0pt}

    \usepackage[Export]{adjustbox} % Used to constrain images to a maximum size
    \adjustboxset{max size={0.9\linewidth}{0.9\paperheight}}
    \usepackage{float}
    \floatplacement{figure}{H} % forces figures to be placed at the correct location
    \usepackage{xcolor} % Allow colors to be defined
    \usepackage{enumerate} % Needed for markdown enumerations to work
    \usepackage{geometry} % Used to adjust the document margins
    \usepackage{amsmath} % Equations
    \usepackage{amssymb} % Equations
    \usepackage{textcomp} % defines textquotesingle
    % Hack from http://tex.stackexchange.com/a/47451/13684:
    \AtBeginDocument{%
        \def\PYZsq{\textquotesingle}% Upright quotes in Pygmentized code
    }
    \usepackage{upquote} % Upright quotes for verbatim code
    \usepackage{eurosym} % defines \euro
    \usepackage[mathletters]{ucs} % Extended unicode (utf-8) support
    \usepackage{fancyvrb} % verbatim replacement that allows latex
    \usepackage{grffile} % extends the file name processing of package graphics 
                         % to support a larger range
    \makeatletter % fix for grffile with XeLaTeX
    \def\Gread@@xetex#1{%
      \IfFileExists{"\Gin@base".bb}%
      {\Gread@eps{\Gin@base.bb}}%
      {\Gread@@xetex@aux#1}%
    }
    \makeatother

    % The hyperref package gives us a pdf with properly built
    % internal navigation ('pdf bookmarks' for the table of contents,
    % internal cross-reference links, web links for URLs, etc.)
    \usepackage{hyperref}
    % The default LaTeX title has an obnoxious amount of whitespace. By default,
    % titling removes some of it. It also provides customization options.
    \usepackage{titling}
    \usepackage{longtable} % longtable support required by pandoc >1.10
    \usepackage{booktabs}  % table support for pandoc > 1.12.2
    \usepackage[inline]{enumitem} % IRkernel/repr support (it uses the enumerate* environment)
    \usepackage[normalem]{ulem} % ulem is needed to support strikethroughs (\sout)
                                % normalem makes italics be italics, not underlines
    \usepackage{mathrsfs}
    

    
    % Colors for the hyperref package
    \definecolor{urlcolor}{rgb}{0,.145,.698}
    \definecolor{linkcolor}{rgb}{.71,0.21,0.01}
    \definecolor{citecolor}{rgb}{.12,.54,.11}

    % ANSI colors
    \definecolor{ansi-black}{HTML}{3E424D}
    \definecolor{ansi-black-intense}{HTML}{282C36}
    \definecolor{ansi-red}{HTML}{E75C58}
    \definecolor{ansi-red-intense}{HTML}{B22B31}
    \definecolor{ansi-green}{HTML}{00A250}
    \definecolor{ansi-green-intense}{HTML}{007427}
    \definecolor{ansi-yellow}{HTML}{DDB62B}
    \definecolor{ansi-yellow-intense}{HTML}{B27D12}
    \definecolor{ansi-blue}{HTML}{208FFB}
    \definecolor{ansi-blue-intense}{HTML}{0065CA}
    \definecolor{ansi-magenta}{HTML}{D160C4}
    \definecolor{ansi-magenta-intense}{HTML}{A03196}
    \definecolor{ansi-cyan}{HTML}{60C6C8}
    \definecolor{ansi-cyan-intense}{HTML}{258F8F}
    \definecolor{ansi-white}{HTML}{C5C1B4}
    \definecolor{ansi-white-intense}{HTML}{A1A6B2}
    \definecolor{ansi-default-inverse-fg}{HTML}{FFFFFF}
    \definecolor{ansi-default-inverse-bg}{HTML}{000000}

    % commands and environments needed by pandoc snippets
    % extracted from the output of `pandoc -s`
    \providecommand{\tightlist}{%
      \setlength{\itemsep}{0pt}\setlength{\parskip}{0pt}}
    \DefineVerbatimEnvironment{Highlighting}{Verbatim}{commandchars=\\\{\}}
    % Add ',fontsize=\small' for more characters per line
    \newenvironment{Shaded}{}{}
    \newcommand{\KeywordTok}[1]{\textcolor[rgb]{0.00,0.44,0.13}{\textbf{{#1}}}}
    \newcommand{\DataTypeTok}[1]{\textcolor[rgb]{0.56,0.13,0.00}{{#1}}}
    \newcommand{\DecValTok}[1]{\textcolor[rgb]{0.25,0.63,0.44}{{#1}}}
    \newcommand{\BaseNTok}[1]{\textcolor[rgb]{0.25,0.63,0.44}{{#1}}}
    \newcommand{\FloatTok}[1]{\textcolor[rgb]{0.25,0.63,0.44}{{#1}}}
    \newcommand{\CharTok}[1]{\textcolor[rgb]{0.25,0.44,0.63}{{#1}}}
    \newcommand{\StringTok}[1]{\textcolor[rgb]{0.25,0.44,0.63}{{#1}}}
    \newcommand{\CommentTok}[1]{\textcolor[rgb]{0.38,0.63,0.69}{\textit{{#1}}}}
    \newcommand{\OtherTok}[1]{\textcolor[rgb]{0.00,0.44,0.13}{{#1}}}
    \newcommand{\AlertTok}[1]{\textcolor[rgb]{1.00,0.00,0.00}{\textbf{{#1}}}}
    \newcommand{\FunctionTok}[1]{\textcolor[rgb]{0.02,0.16,0.49}{{#1}}}
    \newcommand{\RegionMarkerTok}[1]{{#1}}
    \newcommand{\ErrorTok}[1]{\textcolor[rgb]{1.00,0.00,0.00}{\textbf{{#1}}}}
    \newcommand{\NormalTok}[1]{{#1}}
    
    % Additional commands for more recent versions of Pandoc
    \newcommand{\ConstantTok}[1]{\textcolor[rgb]{0.53,0.00,0.00}{{#1}}}
    \newcommand{\SpecialCharTok}[1]{\textcolor[rgb]{0.25,0.44,0.63}{{#1}}}
    \newcommand{\VerbatimStringTok}[1]{\textcolor[rgb]{0.25,0.44,0.63}{{#1}}}
    \newcommand{\SpecialStringTok}[1]{\textcolor[rgb]{0.73,0.40,0.53}{{#1}}}
    \newcommand{\ImportTok}[1]{{#1}}
    \newcommand{\DocumentationTok}[1]{\textcolor[rgb]{0.73,0.13,0.13}{\textit{{#1}}}}
    \newcommand{\AnnotationTok}[1]{\textcolor[rgb]{0.38,0.63,0.69}{\textbf{\textit{{#1}}}}}
    \newcommand{\CommentVarTok}[1]{\textcolor[rgb]{0.38,0.63,0.69}{\textbf{\textit{{#1}}}}}
    \newcommand{\VariableTok}[1]{\textcolor[rgb]{0.10,0.09,0.49}{{#1}}}
    \newcommand{\ControlFlowTok}[1]{\textcolor[rgb]{0.00,0.44,0.13}{\textbf{{#1}}}}
    \newcommand{\OperatorTok}[1]{\textcolor[rgb]{0.40,0.40,0.40}{{#1}}}
    \newcommand{\BuiltInTok}[1]{{#1}}
    \newcommand{\ExtensionTok}[1]{{#1}}
    \newcommand{\PreprocessorTok}[1]{\textcolor[rgb]{0.74,0.48,0.00}{{#1}}}
    \newcommand{\AttributeTok}[1]{\textcolor[rgb]{0.49,0.56,0.16}{{#1}}}
    \newcommand{\InformationTok}[1]{\textcolor[rgb]{0.38,0.63,0.69}{\textbf{\textit{{#1}}}}}
    \newcommand{\WarningTok}[1]{\textcolor[rgb]{0.38,0.63,0.69}{\textbf{\textit{{#1}}}}}
    
    
    % Define a nice break command that doesn't care if a line doesn't already
    % exist.
    \def\br{\hspace*{\fill} \\* }
    % Math Jax compatibility definitions
    \def\gt{>}
    \def\lt{<}
    \let\Oldtex\TeX
    \let\Oldlatex\LaTeX
    \renewcommand{\TeX}{\textrm{\Oldtex}}
    \renewcommand{\LaTeX}{\textrm{\Oldlatex}}
    % Document parameters
    % Document title
    \title{Untitled1}
    
    
    
    
    
% Pygments definitions
\makeatletter
\def\PY@reset{\let\PY@it=\relax \let\PY@bf=\relax%
    \let\PY@ul=\relax \let\PY@tc=\relax%
    \let\PY@bc=\relax \let\PY@ff=\relax}
\def\PY@tok#1{\csname PY@tok@#1\endcsname}
\def\PY@toks#1+{\ifx\relax#1\empty\else%
    \PY@tok{#1}\expandafter\PY@toks\fi}
\def\PY@do#1{\PY@bc{\PY@tc{\PY@ul{%
    \PY@it{\PY@bf{\PY@ff{#1}}}}}}}
\def\PY#1#2{\PY@reset\PY@toks#1+\relax+\PY@do{#2}}

\expandafter\def\csname PY@tok@w\endcsname{\def\PY@tc##1{\textcolor[rgb]{0.73,0.73,0.73}{##1}}}
\expandafter\def\csname PY@tok@c\endcsname{\let\PY@it=\textit\def\PY@tc##1{\textcolor[rgb]{0.25,0.50,0.50}{##1}}}
\expandafter\def\csname PY@tok@cp\endcsname{\def\PY@tc##1{\textcolor[rgb]{0.74,0.48,0.00}{##1}}}
\expandafter\def\csname PY@tok@k\endcsname{\let\PY@bf=\textbf\def\PY@tc##1{\textcolor[rgb]{0.00,0.50,0.00}{##1}}}
\expandafter\def\csname PY@tok@kp\endcsname{\def\PY@tc##1{\textcolor[rgb]{0.00,0.50,0.00}{##1}}}
\expandafter\def\csname PY@tok@kt\endcsname{\def\PY@tc##1{\textcolor[rgb]{0.69,0.00,0.25}{##1}}}
\expandafter\def\csname PY@tok@o\endcsname{\def\PY@tc##1{\textcolor[rgb]{0.40,0.40,0.40}{##1}}}
\expandafter\def\csname PY@tok@ow\endcsname{\let\PY@bf=\textbf\def\PY@tc##1{\textcolor[rgb]{0.67,0.13,1.00}{##1}}}
\expandafter\def\csname PY@tok@nb\endcsname{\def\PY@tc##1{\textcolor[rgb]{0.00,0.50,0.00}{##1}}}
\expandafter\def\csname PY@tok@nf\endcsname{\def\PY@tc##1{\textcolor[rgb]{0.00,0.00,1.00}{##1}}}
\expandafter\def\csname PY@tok@nc\endcsname{\let\PY@bf=\textbf\def\PY@tc##1{\textcolor[rgb]{0.00,0.00,1.00}{##1}}}
\expandafter\def\csname PY@tok@nn\endcsname{\let\PY@bf=\textbf\def\PY@tc##1{\textcolor[rgb]{0.00,0.00,1.00}{##1}}}
\expandafter\def\csname PY@tok@ne\endcsname{\let\PY@bf=\textbf\def\PY@tc##1{\textcolor[rgb]{0.82,0.25,0.23}{##1}}}
\expandafter\def\csname PY@tok@nv\endcsname{\def\PY@tc##1{\textcolor[rgb]{0.10,0.09,0.49}{##1}}}
\expandafter\def\csname PY@tok@no\endcsname{\def\PY@tc##1{\textcolor[rgb]{0.53,0.00,0.00}{##1}}}
\expandafter\def\csname PY@tok@nl\endcsname{\def\PY@tc##1{\textcolor[rgb]{0.63,0.63,0.00}{##1}}}
\expandafter\def\csname PY@tok@ni\endcsname{\let\PY@bf=\textbf\def\PY@tc##1{\textcolor[rgb]{0.60,0.60,0.60}{##1}}}
\expandafter\def\csname PY@tok@na\endcsname{\def\PY@tc##1{\textcolor[rgb]{0.49,0.56,0.16}{##1}}}
\expandafter\def\csname PY@tok@nt\endcsname{\let\PY@bf=\textbf\def\PY@tc##1{\textcolor[rgb]{0.00,0.50,0.00}{##1}}}
\expandafter\def\csname PY@tok@nd\endcsname{\def\PY@tc##1{\textcolor[rgb]{0.67,0.13,1.00}{##1}}}
\expandafter\def\csname PY@tok@s\endcsname{\def\PY@tc##1{\textcolor[rgb]{0.73,0.13,0.13}{##1}}}
\expandafter\def\csname PY@tok@sd\endcsname{\let\PY@it=\textit\def\PY@tc##1{\textcolor[rgb]{0.73,0.13,0.13}{##1}}}
\expandafter\def\csname PY@tok@si\endcsname{\let\PY@bf=\textbf\def\PY@tc##1{\textcolor[rgb]{0.73,0.40,0.53}{##1}}}
\expandafter\def\csname PY@tok@se\endcsname{\let\PY@bf=\textbf\def\PY@tc##1{\textcolor[rgb]{0.73,0.40,0.13}{##1}}}
\expandafter\def\csname PY@tok@sr\endcsname{\def\PY@tc##1{\textcolor[rgb]{0.73,0.40,0.53}{##1}}}
\expandafter\def\csname PY@tok@ss\endcsname{\def\PY@tc##1{\textcolor[rgb]{0.10,0.09,0.49}{##1}}}
\expandafter\def\csname PY@tok@sx\endcsname{\def\PY@tc##1{\textcolor[rgb]{0.00,0.50,0.00}{##1}}}
\expandafter\def\csname PY@tok@m\endcsname{\def\PY@tc##1{\textcolor[rgb]{0.40,0.40,0.40}{##1}}}
\expandafter\def\csname PY@tok@gh\endcsname{\let\PY@bf=\textbf\def\PY@tc##1{\textcolor[rgb]{0.00,0.00,0.50}{##1}}}
\expandafter\def\csname PY@tok@gu\endcsname{\let\PY@bf=\textbf\def\PY@tc##1{\textcolor[rgb]{0.50,0.00,0.50}{##1}}}
\expandafter\def\csname PY@tok@gd\endcsname{\def\PY@tc##1{\textcolor[rgb]{0.63,0.00,0.00}{##1}}}
\expandafter\def\csname PY@tok@gi\endcsname{\def\PY@tc##1{\textcolor[rgb]{0.00,0.63,0.00}{##1}}}
\expandafter\def\csname PY@tok@gr\endcsname{\def\PY@tc##1{\textcolor[rgb]{1.00,0.00,0.00}{##1}}}
\expandafter\def\csname PY@tok@ge\endcsname{\let\PY@it=\textit}
\expandafter\def\csname PY@tok@gs\endcsname{\let\PY@bf=\textbf}
\expandafter\def\csname PY@tok@gp\endcsname{\let\PY@bf=\textbf\def\PY@tc##1{\textcolor[rgb]{0.00,0.00,0.50}{##1}}}
\expandafter\def\csname PY@tok@go\endcsname{\def\PY@tc##1{\textcolor[rgb]{0.53,0.53,0.53}{##1}}}
\expandafter\def\csname PY@tok@gt\endcsname{\def\PY@tc##1{\textcolor[rgb]{0.00,0.27,0.87}{##1}}}
\expandafter\def\csname PY@tok@err\endcsname{\def\PY@bc##1{\setlength{\fboxsep}{0pt}\fcolorbox[rgb]{1.00,0.00,0.00}{1,1,1}{\strut ##1}}}
\expandafter\def\csname PY@tok@kc\endcsname{\let\PY@bf=\textbf\def\PY@tc##1{\textcolor[rgb]{0.00,0.50,0.00}{##1}}}
\expandafter\def\csname PY@tok@kd\endcsname{\let\PY@bf=\textbf\def\PY@tc##1{\textcolor[rgb]{0.00,0.50,0.00}{##1}}}
\expandafter\def\csname PY@tok@kn\endcsname{\let\PY@bf=\textbf\def\PY@tc##1{\textcolor[rgb]{0.00,0.50,0.00}{##1}}}
\expandafter\def\csname PY@tok@kr\endcsname{\let\PY@bf=\textbf\def\PY@tc##1{\textcolor[rgb]{0.00,0.50,0.00}{##1}}}
\expandafter\def\csname PY@tok@bp\endcsname{\def\PY@tc##1{\textcolor[rgb]{0.00,0.50,0.00}{##1}}}
\expandafter\def\csname PY@tok@fm\endcsname{\def\PY@tc##1{\textcolor[rgb]{0.00,0.00,1.00}{##1}}}
\expandafter\def\csname PY@tok@vc\endcsname{\def\PY@tc##1{\textcolor[rgb]{0.10,0.09,0.49}{##1}}}
\expandafter\def\csname PY@tok@vg\endcsname{\def\PY@tc##1{\textcolor[rgb]{0.10,0.09,0.49}{##1}}}
\expandafter\def\csname PY@tok@vi\endcsname{\def\PY@tc##1{\textcolor[rgb]{0.10,0.09,0.49}{##1}}}
\expandafter\def\csname PY@tok@vm\endcsname{\def\PY@tc##1{\textcolor[rgb]{0.10,0.09,0.49}{##1}}}
\expandafter\def\csname PY@tok@sa\endcsname{\def\PY@tc##1{\textcolor[rgb]{0.73,0.13,0.13}{##1}}}
\expandafter\def\csname PY@tok@sb\endcsname{\def\PY@tc##1{\textcolor[rgb]{0.73,0.13,0.13}{##1}}}
\expandafter\def\csname PY@tok@sc\endcsname{\def\PY@tc##1{\textcolor[rgb]{0.73,0.13,0.13}{##1}}}
\expandafter\def\csname PY@tok@dl\endcsname{\def\PY@tc##1{\textcolor[rgb]{0.73,0.13,0.13}{##1}}}
\expandafter\def\csname PY@tok@s2\endcsname{\def\PY@tc##1{\textcolor[rgb]{0.73,0.13,0.13}{##1}}}
\expandafter\def\csname PY@tok@sh\endcsname{\def\PY@tc##1{\textcolor[rgb]{0.73,0.13,0.13}{##1}}}
\expandafter\def\csname PY@tok@s1\endcsname{\def\PY@tc##1{\textcolor[rgb]{0.73,0.13,0.13}{##1}}}
\expandafter\def\csname PY@tok@mb\endcsname{\def\PY@tc##1{\textcolor[rgb]{0.40,0.40,0.40}{##1}}}
\expandafter\def\csname PY@tok@mf\endcsname{\def\PY@tc##1{\textcolor[rgb]{0.40,0.40,0.40}{##1}}}
\expandafter\def\csname PY@tok@mh\endcsname{\def\PY@tc##1{\textcolor[rgb]{0.40,0.40,0.40}{##1}}}
\expandafter\def\csname PY@tok@mi\endcsname{\def\PY@tc##1{\textcolor[rgb]{0.40,0.40,0.40}{##1}}}
\expandafter\def\csname PY@tok@il\endcsname{\def\PY@tc##1{\textcolor[rgb]{0.40,0.40,0.40}{##1}}}
\expandafter\def\csname PY@tok@mo\endcsname{\def\PY@tc##1{\textcolor[rgb]{0.40,0.40,0.40}{##1}}}
\expandafter\def\csname PY@tok@ch\endcsname{\let\PY@it=\textit\def\PY@tc##1{\textcolor[rgb]{0.25,0.50,0.50}{##1}}}
\expandafter\def\csname PY@tok@cm\endcsname{\let\PY@it=\textit\def\PY@tc##1{\textcolor[rgb]{0.25,0.50,0.50}{##1}}}
\expandafter\def\csname PY@tok@cpf\endcsname{\let\PY@it=\textit\def\PY@tc##1{\textcolor[rgb]{0.25,0.50,0.50}{##1}}}
\expandafter\def\csname PY@tok@c1\endcsname{\let\PY@it=\textit\def\PY@tc##1{\textcolor[rgb]{0.25,0.50,0.50}{##1}}}
\expandafter\def\csname PY@tok@cs\endcsname{\let\PY@it=\textit\def\PY@tc##1{\textcolor[rgb]{0.25,0.50,0.50}{##1}}}

\def\PYZbs{\char`\\}
\def\PYZus{\char`\_}
\def\PYZob{\char`\{}
\def\PYZcb{\char`\}}
\def\PYZca{\char`\^}
\def\PYZam{\char`\&}
\def\PYZlt{\char`\<}
\def\PYZgt{\char`\>}
\def\PYZsh{\char`\#}
\def\PYZpc{\char`\%}
\def\PYZdl{\char`\$}
\def\PYZhy{\char`\-}
\def\PYZsq{\char`\'}
\def\PYZdq{\char`\"}
\def\PYZti{\char`\~}
% for compatibility with earlier versions
\def\PYZat{@}
\def\PYZlb{[}
\def\PYZrb{]}
\makeatother


    % For linebreaks inside Verbatim environment from package fancyvrb. 
    \makeatletter
        \newbox\Wrappedcontinuationbox 
        \newbox\Wrappedvisiblespacebox 
        \newcommand*\Wrappedvisiblespace {\textcolor{red}{\textvisiblespace}} 
        \newcommand*\Wrappedcontinuationsymbol {\textcolor{red}{\llap{\tiny$\m@th\hookrightarrow$}}} 
        \newcommand*\Wrappedcontinuationindent {3ex } 
        \newcommand*\Wrappedafterbreak {\kern\Wrappedcontinuationindent\copy\Wrappedcontinuationbox} 
        % Take advantage of the already applied Pygments mark-up to insert 
        % potential linebreaks for TeX processing. 
        %        {, <, #, %, $, ' and ": go to next line. 
        %        _, }, ^, &, >, - and ~: stay at end of broken line. 
        % Use of \textquotesingle for straight quote. 
        \newcommand*\Wrappedbreaksatspecials {% 
            \def\PYGZus{\discretionary{\char`\_}{\Wrappedafterbreak}{\char`\_}}% 
            \def\PYGZob{\discretionary{}{\Wrappedafterbreak\char`\{}{\char`\{}}% 
            \def\PYGZcb{\discretionary{\char`\}}{\Wrappedafterbreak}{\char`\}}}% 
            \def\PYGZca{\discretionary{\char`\^}{\Wrappedafterbreak}{\char`\^}}% 
            \def\PYGZam{\discretionary{\char`\&}{\Wrappedafterbreak}{\char`\&}}% 
            \def\PYGZlt{\discretionary{}{\Wrappedafterbreak\char`\<}{\char`\<}}% 
            \def\PYGZgt{\discretionary{\char`\>}{\Wrappedafterbreak}{\char`\>}}% 
            \def\PYGZsh{\discretionary{}{\Wrappedafterbreak\char`\#}{\char`\#}}% 
            \def\PYGZpc{\discretionary{}{\Wrappedafterbreak\char`\%}{\char`\%}}% 
            \def\PYGZdl{\discretionary{}{\Wrappedafterbreak\char`\$}{\char`\$}}% 
            \def\PYGZhy{\discretionary{\char`\-}{\Wrappedafterbreak}{\char`\-}}% 
            \def\PYGZsq{\discretionary{}{\Wrappedafterbreak\textquotesingle}{\textquotesingle}}% 
            \def\PYGZdq{\discretionary{}{\Wrappedafterbreak\char`\"}{\char`\"}}% 
            \def\PYGZti{\discretionary{\char`\~}{\Wrappedafterbreak}{\char`\~}}% 
        } 
        % Some characters . , ; ? ! / are not pygmentized. 
        % This macro makes them "active" and they will insert potential linebreaks 
        \newcommand*\Wrappedbreaksatpunct {% 
            \lccode`\~`\.\lowercase{\def~}{\discretionary{\hbox{\char`\.}}{\Wrappedafterbreak}{\hbox{\char`\.}}}% 
            \lccode`\~`\,\lowercase{\def~}{\discretionary{\hbox{\char`\,}}{\Wrappedafterbreak}{\hbox{\char`\,}}}% 
            \lccode`\~`\;\lowercase{\def~}{\discretionary{\hbox{\char`\;}}{\Wrappedafterbreak}{\hbox{\char`\;}}}% 
            \lccode`\~`\:\lowercase{\def~}{\discretionary{\hbox{\char`\:}}{\Wrappedafterbreak}{\hbox{\char`\:}}}% 
            \lccode`\~`\?\lowercase{\def~}{\discretionary{\hbox{\char`\?}}{\Wrappedafterbreak}{\hbox{\char`\?}}}% 
            \lccode`\~`\!\lowercase{\def~}{\discretionary{\hbox{\char`\!}}{\Wrappedafterbreak}{\hbox{\char`\!}}}% 
            \lccode`\~`\/\lowercase{\def~}{\discretionary{\hbox{\char`\/}}{\Wrappedafterbreak}{\hbox{\char`\/}}}% 
            \catcode`\.\active
            \catcode`\,\active 
            \catcode`\;\active
            \catcode`\:\active
            \catcode`\?\active
            \catcode`\!\active
            \catcode`\/\active 
            \lccode`\~`\~ 	
        }
    \makeatother

    \let\OriginalVerbatim=\Verbatim
    \makeatletter
    \renewcommand{\Verbatim}[1][1]{%
        %\parskip\z@skip
        \sbox\Wrappedcontinuationbox {\Wrappedcontinuationsymbol}%
        \sbox\Wrappedvisiblespacebox {\FV@SetupFont\Wrappedvisiblespace}%
        \def\FancyVerbFormatLine ##1{\hsize\linewidth
            \vtop{\raggedright\hyphenpenalty\z@\exhyphenpenalty\z@
                \doublehyphendemerits\z@\finalhyphendemerits\z@
                \strut ##1\strut}%
        }%
        % If the linebreak is at a space, the latter will be displayed as visible
        % space at end of first line, and a continuation symbol starts next line.
        % Stretch/shrink are however usually zero for typewriter font.
        \def\FV@Space {%
            \nobreak\hskip\z@ plus\fontdimen3\font minus\fontdimen4\font
            \discretionary{\copy\Wrappedvisiblespacebox}{\Wrappedafterbreak}
            {\kern\fontdimen2\font}%
        }%
        
        % Allow breaks at special characters using \PYG... macros.
        \Wrappedbreaksatspecials
        % Breaks at punctuation characters . , ; ? ! and / need catcode=\active 	
        \OriginalVerbatim[#1,codes*=\Wrappedbreaksatpunct]%
    }
    \makeatother

    % Exact colors from NB
    \definecolor{incolor}{HTML}{303F9F}
    \definecolor{outcolor}{HTML}{D84315}
    \definecolor{cellborder}{HTML}{CFCFCF}
    \definecolor{cellbackground}{HTML}{F7F7F7}
    
    % prompt
    \makeatletter
    \newcommand{\boxspacing}{\kern\kvtcb@left@rule\kern\kvtcb@boxsep}
    \makeatother
    \newcommand{\prompt}[4]{
        \ttfamily\llap{{\color{#2}[#3]:\hspace{3pt}#4}}\vspace{-\baselineskip}
    }
    

    
    % Prevent overflowing lines due to hard-to-break entities
    \sloppy 
    % Setup hyperref package
    \hypersetup{
      breaklinks=true,  % so long urls are correctly broken across lines
      colorlinks=true,
      urlcolor=urlcolor,
      linkcolor=linkcolor,
      citecolor=citecolor,
      }
    % Slightly bigger margins than the latex defaults
    
    \geometry{verbose,tmargin=1in,bmargin=1in,lmargin=1in,rmargin=1in}

\begin{document}
\begin{titlepage}
	\centering
	%\includegraphics[width=0.15\textwidth]{example-image-1x1}\par\vspace{1cm}
	{\scshape\LARGE Indian Institute of Technology\\ Jammu \par}
	\vspace{4cm}
	{\huge\bfseries Moth Flame Optimisation Algorithm\par}
	\vspace{2cm}
	{\Large\itshape Sahil Jindal\par}
	{2016UEE0042 }
	\vfill
	submitted to\par
	Dr.~Yamuna Prasad \par
	Course : Natural language Processing

	\vfill

% Bottom of the page
	{\large May 20,2020\par}
\end{titlepage}



\newpage
\section*{Summary}
\vspace{2cm}
\subsection*{I.\hspace{5pt}	 Main Idea}

The main inspiration of this optimizer is the navigation method of moths in nature called transverse orientation, where the moths move while maintaining a fixed angle to the light source.

\subsection*{II.\hspace{5pt}	Nature and Behavior of Moths}

Moths fly in night by maintaining a fixed angle with respect to the moon, a very effective mechanism for travelling in a straight line for long distances. These insects are trapped in a useless/deadly spiral path around artificial lights. This paper mathematically models this behavior to perform optimization.
\subsection*{III.\hspace{5pt}	Moth Flame Optimization Summary}
\begin{itemize}

\item In the proposed MFO algorithm, it is assumed that the candidate solutions are moths and the problems variables are the position of moths in the space. 

\item •	This optimization algorithm works like particle swarm optimization and genetic algorithm.

\item Since the Moth Flame Optimisation algorithm is a population-based algorithm, the set of moths is represented in a matrix M where n is the number of moths and d is the number of variables.

$$
A =
\begin{bmatrix} 
m_{1,1} & m_{1,2} & \dots & \dots & m_{1,d} \\
m_{2,1} & m_{2,2} & \dots & \dots & m_{2,d}\\
\vdots  & \vdots  & \vdots & \vdots &\vdots\\
m_{n,1} & m_{n,2} & \dots & \dots & m_{n,d} \\
\end{bmatrix}
\quad
$$

\item For all the moths, we also assume that there is an array for storing the corresponding fitness values OM where n is the number of moths.

$$
OM = 
\begin{bmatrix} 
OM_{1}\\
OM_{2}\\
\vdots\\
OM_{n}\\
\end{bmatrix}
\quad
$$

\item Another key component in the proposed algorithm are flames. 

\item A matrix like the moth matrix is considered as F which has same dimension as M and has an array for storing corresponding fitness values in a matrix OF.

$$
F =
\begin{bmatrix} 
F_{1,1} & F_{1,2} & \dots & \dots & F_{1,d} \\
F_{2,1} & F_{2,2} & \dots & \dots & F_{2,d}\\
\vdots  & \vdots  & \vdots & \vdots &\vdots\\
F_{n,1} & F_{n,2} & \dots & \dots & F_{n,d} \\
\end{bmatrix}
\quad
$$
\vspace{25pt}
$$
OF = 
\begin{bmatrix} 
OF_{1}\\
OF_{2}\\
\vdots\\
OF_{n}\\
\end{bmatrix}
\quad
$$

\item Here the moths and the flames are both solutions, the only difference being the way they are iterated. 

\item The moths are actual search agents that move around the search space, whereas flames are the best position of moths that obtains so far. 

\item Hence, each moth searches around the flame and updates it if it finds a better solution. Thus, a moth never loses his best solution.

\item Each moth is obliged to update its position using only one of the flames.

\item The reason that why a specific flame is assigned to each moth is to prevent local optimum stagnation.

\item If all the moths get attracted to a single flame, all of them converge to a point in the search spaces because they can only fly towards a flame and not outwards, requiring them to move around different flames, however, causes higher exploration of the search space and lower probability of local optima stagnation. 

\item A logarithmic spiral is chosen as the main update mechanism of moths, but any spiral can be chosen if the moth is the starting point and the flame is the ending point of the spiral and also the spiral should not leave the search space. 

\item A logarithmic spiral is defined as follows where Di indicates the distance of the $i_{th}$ moth for the $j_{th}$ flame, b is a constant for defining the shape of the logarithmic spiral, and t is a random number in [-1,1]. D is calculated as $\abs{F_{j} - M_{i}}$ where $M_{i}$ is the $i_{th}$ moth, $F_{j}$ is the $j_{th}$ flame, and $D_{i}$ is the distance of the $i_{th}$ moth from the $j_{th}$ flame.

$$
S(M_{i},F_{j}) = D_{i} \cdot e^{bt} \cdot \sin{2 \pi t} + F_{j}
$$

\item The t parameter in the spiral equation shows how much the next position of the moth should be close to the flame as $t = -1$ is the closest and $t = 1 $ is the farthest position from the flame. 

\item The spiral equation allows a moth to fly around a flame and not necessarily in the space between them. To avoid the degradation of exploration, the following formula is used- 

$$
flame_no = round(N - l \ast \frac{N - 1}{T}) 
$$

\item Here, I is the current number of iterations, N is maximum number of flames and T is the maximum number of iterations.

\end{itemize}

\textbf{Complexity of algorithm is O(n)} 

\subsection*{IV.\hspace{5pt}	Algorithm}

\begin{itemize}
\item The algorithm starts by initializing the population($x_{i}$) with a random lower bound and upper bound.
\item Sort $x_{i}$ and set $x_{i}$ as [ $x_{i}$ ] matrix.
\item Calculate the fitness function of each search agent.
\item Calculate the best moth and the best fitness.
\item Calculate the fitness function and tag the best position by the flames
\item Update flame number, t and r.
\item Calculate D for the corresponding moth.
\item Update $ M_{i,j}$ for the corresponding moth.
\item If the termination criteria are satisfied, report the best position of the moth.
\end{itemize}


\subsection*{V.\hspace{5pt}	Code}
\vspace{30pt}
    
    

    
    \begin{tcolorbox}[breakable, size=fbox, boxrule=1pt, pad at break*=1mm,colback=cellbackground, colframe=cellborder]

\begin{Verbatim}[commandchars=\\\{\}]
\PY{k+kn}{import} \PY{n+nn}{numpy} \PY{k}{as} \PY{n+nn}{np}
\PY{k+kn}{import} \PY{n+nn}{matplotlib}\PY{n+nn}{.}\PY{n+nn}{pyplot} \PY{k}{as} \PY{n+nn}{plt}

\PY{k}{def} \PY{n+nf}{F1}\PY{p}{(}\PY{n}{x}\PY{p}{)}\PY{p}{:} \PY{c+c1}{\PYZsh{} fitness function}
  \PY{k}{return} \PY{n}{np}\PY{o}{.}\PY{n}{sum}\PY{p}{(}\PY{n}{np}\PY{o}{.}\PY{n}{power}\PY{p}{(}\PY{n}{x}\PY{p}{,} \PY{l+m+mi}{2}\PY{p}{)}\PY{p}{,} \PY{n}{axis}\PY{o}{=}\PY{l+m+mi}{1}\PY{p}{)}

\PY{k}{def} \PY{n+nf}{MFO}\PY{p}{(}\PY{n}{nsa}\PY{p}{,} \PY{n}{dim}\PY{p}{,} \PY{n}{ub}\PY{p}{,} \PY{n}{lb}\PY{p}{,} \PY{n}{max\PYZus{}iter}\PY{p}{,} \PY{n}{fobj}\PY{p}{)}\PY{p}{:}

  \PY{c+c1}{\PYZsh{} Initialize the positions of moths}
  \PY{n}{mothPos} \PY{o}{=} \PY{n}{np}\PY{o}{.}\PY{n}{random}\PY{o}{.}\PY{n}{uniform}\PY{p}{(}\PY{n}{low}\PY{o}{=}\PY{n}{lb}\PY{p}{,} \PY{n}{high}\PY{o}{=}\PY{n}{ub}\PY{p}{,} \PY{n}{size}\PY{o}{=}\PY{p}{(}\PY{n}{nsa}\PY{p}{,} \PY{n}{dim}\PY{p}{)}\PY{p}{)}

  \PY{n}{convergenceCurve} \PY{o}{=} \PY{n}{np}\PY{o}{.}\PY{n}{zeros}\PY{p}{(}\PY{n}{shape}\PY{o}{=}\PY{p}{(}\PY{n}{max\PYZus{}iter}\PY{p}{)}\PY{p}{)}

  
  \PY{k}{for} \PY{n}{iteration} \PY{o+ow}{in} \PY{n+nb}{range}\PY{p}{(}\PY{n}{max\PYZus{}iter}\PY{p}{)}\PY{p}{:}  
    \PY{n}{flameNo} \PY{o}{=} \PY{n+nb}{int}\PY{p}{(}\PY{n}{np}\PY{o}{.}\PY{n}{ceil}\PY{p}{(}\PY{n}{nsa}\PY{o}{\PYZhy{}}\PY{p}{(}\PY{n}{iteration}\PY{o}{+}\PY{l+m+mi}{1}\PY{p}{)}\PY{o}{*}\PY{p}{(}\PY{p}{(}\PY{n}{nsa}\PY{o}{\PYZhy{}}\PY{l+m+mi}{1}\PY{p}{)}\PY{o}{/}\PY{n}{max\PYZus{}iter}\PY{p}{)}\PY{p}{)}\PY{p}{)}
    \PY{n}{mothPos} \PY{o}{=} \PY{n}{np}\PY{o}{.}\PY{n}{clip}\PY{p}{(}\PY{n}{mothPos}\PY{p}{,} \PY{n}{lb}\PY{p}{,} \PY{n}{ub}\PY{p}{)}
    \PY{n}{mothFit} \PY{o}{=} \PY{n}{fobj}\PY{p}{(}\PY{n}{mothPos}\PY{p}{)}
    \PY{k}{if} \PY{n}{iteration} \PY{o}{==} \PY{l+m+mi}{0}\PY{p}{:}
      \PY{n}{order} \PY{o}{=} \PY{n}{mothFit}\PY{o}{.}\PY{n}{argsort}\PY{p}{(}\PY{p}{)}
      \PY{n}{mothFit} \PY{o}{=} \PY{n}{mothFit}\PY{p}{[}\PY{n}{order}\PY{p}{]}
      \PY{n}{mothPos} \PY{o}{=} \PY{n}{mothPos}\PY{p}{[}\PY{n}{order}\PY{p}{,} \PY{p}{:}\PY{p}{]}
      \PY{n}{bFlames} \PY{o}{=} \PY{n}{np}\PY{o}{.}\PY{n}{copy}\PY{p}{(}\PY{n}{mothPos}\PY{p}{)}
      \PY{n}{bFlamesFit} \PY{o}{=} \PY{n}{np}\PY{o}{.}\PY{n}{copy}\PY{p}{(}\PY{n}{mothFit}\PY{p}{)}

    \PY{k}{else}\PY{p}{:}
      \PY{n}{doublePop} \PY{o}{=} \PY{n}{np}\PY{o}{.}\PY{n}{vstack}\PY{p}{(}\PY{p}{(}\PY{n}{bFlames}\PY{p}{,} \PY{n}{mothPos}\PY{p}{)}\PY{p}{)}
      \PY{n}{doubleFit} \PY{o}{=} \PY{n}{np}\PY{o}{.}\PY{n}{hstack}\PY{p}{(}\PY{p}{(}\PY{n}{bFlamesFit}\PY{p}{,} \PY{n}{mothFit}\PY{p}{)}\PY{p}{)}
      \PY{n}{order} \PY{o}{=} \PY{n}{doubleFit}\PY{o}{.}\PY{n}{argsort}\PY{p}{(}\PY{p}{)}
      \PY{n}{doubleFit} \PY{o}{=} \PY{n}{doubleFit}\PY{p}{[}\PY{n}{order}\PY{p}{]}
      \PY{n}{doublePop} \PY{o}{=} \PY{n}{doublePop}\PY{p}{[}\PY{n}{order}\PY{p}{,} \PY{p}{:}\PY{p}{]}
      \PY{n}{bFlames} \PY{o}{=} \PY{n}{doublePop}\PY{p}{[}\PY{p}{:}\PY{n}{nsa}\PY{p}{,} \PY{p}{:}\PY{p}{]}
      \PY{n}{bFlamesFit} \PY{o}{=} \PY{n}{doubleFit}\PY{p}{[}\PY{p}{:}\PY{n}{nsa}\PY{p}{]}

    \PY{n}{bFlameScore} \PY{o}{=} \PY{n}{bFlamesFit}\PY{p}{[}\PY{l+m+mi}{0}\PY{p}{]}
    \PY{n}{bFlamesPos} \PY{o}{=} \PY{n}{bFlames}\PY{p}{[}\PY{l+m+mi}{0}\PY{p}{,} \PY{p}{:}\PY{p}{]}
    \PY{n}{a} \PY{o}{=} \PY{o}{\PYZhy{}}\PY{l+m+mi}{1} \PY{o}{+} \PY{p}{(}\PY{n}{iteration}\PY{o}{+}\PY{l+m+mi}{1}\PY{p}{)} \PY{o}{*} \PY{p}{(}\PY{p}{(}\PY{o}{\PYZhy{}}\PY{l+m+mi}{1}\PY{p}{)}\PY{o}{/}\PY{n}{max\PYZus{}iter}\PY{p}{)}
    \PY{n}{distanceToFlames} \PY{o}{=} \PY{n}{np}\PY{o}{.}\PY{n}{abs}\PY{p}{(}\PY{n}{bFlames} \PY{o}{\PYZhy{}} \PY{n}{mothPos}\PY{p}{)}

    \PY{n}{b} \PY{o}{=} \PY{l+m+mi}{1}
    \PY{n}{t} \PY{o}{=} \PY{p}{(}\PY{n}{a}\PY{o}{\PYZhy{}}\PY{l+m+mi}{1}\PY{p}{)}\PY{o}{*}\PY{n}{np}\PY{o}{.}\PY{n}{random}\PY{o}{.}\PY{n}{rand}\PY{p}{(}\PY{n}{nsa}\PY{p}{,} \PY{n}{dim}\PY{p}{)} \PY{o}{+} \PY{l+m+mi}{1}
    \PY{n}{temp1} \PY{o}{=} \PY{n}{bFlames}\PY{p}{[}\PY{p}{:}\PY{n}{flameNo}\PY{p}{,} \PY{p}{:}\PY{p}{]}
    \PY{n}{temp2} \PY{o}{=} \PY{n}{bFlames}\PY{p}{[}\PY{n}{flameNo}\PY{o}{\PYZhy{}}\PY{l+m+mi}{1}\PY{p}{,} \PY{p}{:}\PY{p}{]}\PY{o}{*}\PY{n}{np}\PY{o}{.}\PY{n}{ones}\PY{p}{(}\PY{n}{shape}\PY{o}{=}\PY{p}{(}\PY{n}{nsa}\PY{o}{\PYZhy{}}\PY{n}{flameNo}\PY{p}{,} \PY{n}{dim}\PY{p}{)}\PY{p}{)}
    \PY{n}{temp2} \PY{o}{=} \PY{n}{np}\PY{o}{.}\PY{n}{vstack}\PY{p}{(}\PY{p}{(}\PY{n}{temp1}\PY{p}{,} \PY{n}{temp2}\PY{p}{)}\PY{p}{)}
    \PY{n}{mothPos} \PY{o}{=} \PY{n}{distanceToFlames}\PY{o}{*}\PY{n}{np}\PY{o}{.}\PY{n}{exp}\PY{p}{(}\PY{n}{b}\PY{o}{*}\PY{n}{t}\PY{p}{)}\PY{o}{*}\PY{n}{np}\PY{o}{.}\PY{n}{cos}\PY{p}{(}\PY{n}{t}\PY{o}{*}\PY{l+m+mi}{2}\PY{o}{*}\PY{n}{np}\PY{o}{.}\PY{n}{pi}\PY{p}{)} \PY{o}{+} \PY{n}{temp2}

    \PY{n}{convergenceCurve}\PY{p}{[}\PY{n}{iteration}\PY{p}{]} \PY{o}{=} \PY{n}{bFlameScore}
    \PY{k}{if} \PY{n}{iteration} \PY{o}{\PYZpc{}}\PY{k}{100} == 0:
        \PY{n+nb}{print}\PY{p}{(}\PY{l+s+s2}{\PYZdq{}}\PY{l+s+s2}{Iteration = }\PY{l+s+s2}{\PYZdq{}}\PY{p}{,} \PY{n}{iteration}\PY{p}{)}
        \PY{n+nb}{print}\PY{p}{(}\PY{l+s+s2}{\PYZdq{}}\PY{l+s+s2}{Best Flame Score = }\PY{l+s+s2}{\PYZdq{}}\PY{p}{,} \PY{n}{bFlameScore}\PY{p}{)}

  \PY{k}{return} \PY{n}{bFlameScore}\PY{p}{,} \PY{n}{bFlamesPos}\PY{p}{,} \PY{n}{convergenceCurve}

\PY{n}{nsa} \PY{o}{=} \PY{l+m+mi}{10}
\PY{n}{max\PYZus{}iter} \PY{o}{=} \PY{l+m+mi}{1001}

\PY{n}{lb} \PY{o}{=} \PY{o}{\PYZhy{}}\PY{l+m+mi}{100}
\PY{n}{ub} \PY{o}{=} \PY{l+m+mi}{100}
\PY{n}{dim} \PY{o}{=} \PY{l+m+mi}{10}

\PY{n}{bFlameScore}\PY{p}{,} \PY{n}{bFlamesPos}\PY{p}{,} \PY{n}{convergenceCurve} \PY{o}{=} \PY{n}{MFO}\PY{p}{(}\PY{n}{nsa}\PY{p}{,} \PY{n}{dim}\PY{p}{,} \PY{n}{ub}\PY{p}{,} \PY{n}{lb}\PY{p}{,} \PY{n}{max\PYZus{}iter}\PY{p}{,} \PY{n}{F1}\PY{p}{)}

\PY{n+nb}{print}\PY{p}{(}\PY{l+s+s2}{\PYZdq{}}\PY{l+s+s2}{Final flames Score = }\PY{l+s+s2}{\PYZdq{}}\PY{p}{,} \PY{n}{bFlameScore}\PY{p}{)}
\PY{n+nb}{print}\PY{p}{(}\PY{l+s+s2}{\PYZdq{}}\PY{l+s+s2}{Final flames position = }\PY{l+s+s2}{\PYZdq{}}\PY{p}{,} \PY{n}{bFlamesPos}\PY{p}{)}
\PY{n}{x} \PY{o}{=} \PY{n}{np}\PY{o}{.}\PY{n}{arange}\PY{p}{(}\PY{l+m+mi}{0}\PY{p}{,} \PY{n}{max\PYZus{}iter}\PY{p}{,} \PY{l+m+mi}{1}\PY{p}{)}
\end{Verbatim}
\end{tcolorbox}

\subsection*{VI.\hspace{5pt}	Output}

\vspace{30pt}


    \begin{Verbatim}[commandchars=\\\{\}]
Iteration =  0
Best Flame Score =  25743.307095807824
Iteration =  100
Best Flame Score =  54.24248747378081
Iteration =  200
Best Flame Score =  5.195658575081754
Iteration =  300
Best Flame Score =  0.007602338536858343
Iteration =  400
Best Flame Score =  1.1778773993607353e-05
Iteration =  500
Best Flame Score =  2.3317591899468814e-07
Iteration =  600
Best Flame Score =  2.610348257728533e-09
Iteration =  700
Best Flame Score =  1.1019255825451095e-10
Iteration =  800
Best Flame Score =  1.0322801181574019e-10
Iteration =  900
Best Flame Score =  9.981675892074768e-11
Iteration =  1000
Best Flame Score =  9.981675886010315e-11
Final flames Score =  9.981675886010315e-11
Final flames position =  [ 2.61861220e-08 -9.98209190e-06  8.72711532e-09
-1.33211920e-07
 -1.23964067e-07  6.18037449e-08  2.31980556e-08  3.34705436e-07
  5.77463034e-08  1.44933083e-07]
    \end{Verbatim}


    % Add a bibliography block to the postdoc
    
    



\end{document}